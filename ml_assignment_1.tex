%Template
% !TeX spellcheck = de 
\documentclass[a4paper]{scrartcl}
\usepackage[utf8]{inputenc}
%\usepackage[ngerman]{babel}
\usepackage{geometry,forloop,fancyhdr,fancybox,lastpage}
\usepackage{listings}
\lstset{frame=tb,
	language=Java,
	aboveskip=3mm,
	belowskip=3mm,
	showstringspaces=false,
	columns=flexible,
	basicstyle={\small\ttfamily},
	numbers=left,
	numberstyle=\tiny\color{gray},
	keywordstyle=\color{blue},
	commentstyle=\color{dkgreen},
	stringstyle=\color{mauve},
	breaklines=true,
	breakatwhitespace=true,
	tabsize=3
}
\geometry{a4paper,left=3cm, right=3cm, top=3cm, bottom=3cm}
% Diese Daten müssen pro Blatt angepasst werden:
\newcommand{\NUMBER}{6}
\newcommand{\EXERCISES}{3}
% Diese Daten müssen einmalig pro Vorlesung angepasst werden:
\newcommand{\COURSE}{Einführung in Maschinelles Lernen}
\newcommand{\TUTOR}{TBD}
\newcommand{\STUDENTA}{Maria Heitmeier}
\newcommand{\STUDENTB}{Gwent Krause}
\newcommand{\STUDENTC}{Stefan Wezel}
\newcommand{\DEADLINE}{07.06.2018}




%Math
\usepackage{amsmath,amssymb,tabularx}

%Figures
\usepackage{graphicx,tikz,color,float}
\graphicspath{ {assignment1_data/plots/} }
\usetikzlibrary{shapes,trees}

%Algorithms
\usepackage[ruled,linesnumbered]{algorithm2e}

%Kopf- und Fußzeile
\pagestyle {fancy}
\fancyhead[L]{\COURSE}
\fancyhead[C]{\STUDENTA, \STUDENTB, \STUDENTC\\}
\fancyhead[R]{\today}

\fancyfoot[L]{}
\fancyfoot[C]{}
\fancyfoot[R]{Seite \thepage}

%Formatierung der Überschrift, hier nichts ändern
\def\header#1#2{
	\begin{center}
		{\Large\bf Übungsblatt #1}\\
		{(Abgabetermin #2)}
	\end{center}
}

%Definition der Punktetabelle, hier nichts ändern
\newcounter{punktelistectr}
\newcounter{punkte}
\newcommand{\punkteliste}[2]{%
	\setcounter{punkte}{#2}%
	\addtocounter{punkte}{-#1}%
	\stepcounter{punkte}%<-- also punkte = m-n+1 = Anzahl Spalten[1]
	\begin{center}%
		\begin{tabularx}{\linewidth}[]{@{}*{\thepunkte}{>{\centering\arraybackslash} X|}@{}>{\centering\arraybackslash}X}
			\forloop{punktelistectr}{#1}{\value{punktelistectr} < #2 } %
			{%
				\thepunktelistectr &
			}
			#2 &  $\Sigma$ \\
			\hline
			\forloop{punktelistectr}{#1}{\value{punktelistectr} < #2 } %
			{%
				&	
			} &\\
			\forloop{punktelistectr}{#1}{\value{punktelistectr} < #2 } %
			{%
				&
			} &\\
		\end{tabularx}
	\end{center}
}

\begin{document}
	
	
\section*{Aufgabe 1}
\subsection*{(a)}
Bei Klassifikation wird anhand von Features einem Sample eine Kategorie zugeteilt. Bei Regression hingegen wird eine Eigenschaft anhand von anderen (korrelierenden) Eigenschaften eines Samples vorausgesagt.

\subsection*{(b)}
Es handelt sich um ein Regressionsproblem, da ein Feature (Groesse) eines Samples(Mensch) anhand von korrellierenden Features vorhergesagt.

\subsection*{(c)}
Da man dem Sample (Person) eine Kategorie (Geschlecht) zuteilt handelt es sich um ein Klassifizierungsproblem.


\subsection*{(d)}
Beim Supervised Learning benutzt man Daten mit Labels und versucht dann üblicherweise diese für Daten mit unbekannten Labels vorauszusagen.\\
Beim Unsupervised Learning haben die Daten keine Labels und man versucht stattdessen in verschiedene Kategorien einzuteilen (bspw. durch Clustering).

\subsection*{(e)}
\subsubsection*{(i)}
Eine Zufallsvariable ist eine Funktion die aus einer Menge von möglichen Ergebnissen eines Zufallsexperiment auf eine Wahrscheinlichkeit (Maß) abbildet. Hat man eine endliche Augangsmenge von möglichen Ergebnissen (zb. bei einem Würfel) hat man eine diskrete Zufallsvariable. Ist die Grundmenge nicht abzählbar (zb. Körpergröße) dann ist die Zufallsvariable ein Intervall aus einer kontinuierlichen Menge.

\subsubsection*{(ii)}
Eine Wahrscheinlichkeitsfunktion nimmt als Parameter eine diskrete Zufallsvariable an und gibt einen Wert zwischen 0 und 1 aus.\\
Bei der Wahrscheinlickeitsdichtefunktion hingegen ist die Wahrscheinlichkeit für einen bestimmten Wert immer 0 und es lässt sich lediglich die Wahrscheinlichkeit für Intervalle bestimmen. Diese sind dann das Integral der Dichtefunktion welches als Grenzen die obere und untere Schranke des Intervalls hat.

\subsubsection*{(iii)} 
Der Erwartungswert ist der Wert den eine Zufallsvariable nach unendlich vielen Durchführungen eines Zufallsexperiments im Mittel annimmt.

\subsubsection*{(iv)} 
Die Varianz einer Verteilung beschreibt die mittlere quadratische Abweichung zum Erwartungswert. Die Standardabweichung ist die positive Wurzel aus der Varianz. Beide beschreiben also die "Streuung" der Daten um einen durchschnittlichen Wert.

\subsubsection*{(v)} %TODO Formulierung ueberarbeiten.
Die Kovarianz zweier Zufallsvariablen beschreibt wie stark diese linear zusammenhängen. Wird beispielsweise Variable $X$ größer und $Y$ ebenfalls, so spricht man von positiver Kovarianz. (Fall $COV(X,Y)> 0$). Ist das Gegenteil der Fall, also wenn $X$ größer wird und $Y$ kleiner, so spricht man von negativer Kovarianz(Fall $COV(X,Y) < 0$). Wenn $COV(X,Y)=0$, dann sind die Variablen unabhängig.



\subsection*{(f)}
Die Multiplikation zweier Wahrscheinlichkeitsdichten liefert wieder eine Wahrscheinlichkeitsdichte. Wahrscheinlichkeitsdichten lassen sich als Funktion beschreiben, die:
\begin{itemize}
	\item niemals kleiner 0 wird.
	\item unter deren Kurve die Fläche 1 ist (und damit kontinuierlich ist.)
\end{itemize}
Multipliziert man nun zwei Wahrscheinlichkeitsdichten ist das Produkt wieder eine Funktion, deren Werte niemals kleiner 0 werden, und da Wahrscheinlichkeiten immer $\leq 1$ sind, ist die Multiplikation zweier Wahrscheinlichkeiten auch immer $\leq 1$. 

\newpage

\section*{Aufgabe 2}
\subsection*{(a)}
\begin{center}
	\includegraphics*[scale = 0.5]{question2a.png}
\end{center}
\subsection*{(b)}
$\mu = -2:$
\begin{center}
	\includegraphics*[scale = 0.5]{question2b1.png}
\end{center}
Durch das Aendern von $\mu$ zu -2 wird die Kurve um 2 nach links verschoben.\\ \ \\
$\sigma = 2:$
\begin{center}
	\includegraphics*[scale = 0.5]{question2b2.png}
\end{center}
Durch das Ändern von $\sigma$ zu 2 wird die Kurve um Faktor 2 nach außen gedehnt. Somit ist auch das Maximum nur noch halb so groß wie vorher.\\

\subsection*{(c)}
Um die Wahrscheinlichkeit für ein Intervall zu bestimmen kann man die Funktion $f$ integrieren und als Grenzen des Integrals die Grenzen des Intervalls benutzen. Das Integral, welches dann die Fläche unter der Dichtefunktion in diesem Intervall liefert ist dann die Wahrscheinlichkeit.\\
Für unsere gegebene Dichtefunktion ergibt sich für das Intervall $\left[-1,1\right]$ die gerundete Wahrscheinlichkeit $0.683$.


\subsection*{(d)}
Da die Wahrscheinlichkeit an einem Punkt $x$ ohnehin 0 ist, kann die Dichtefunktion an diesem Punkt durchaus einen Wert größer 1 annehmen. Die Wahrscheinlichkeit ist ja die Fläche unter diesem Punkt und die ist immer 0. Die Bedingung ist lediglich, dass die Flaeche unter einem Intervall (welches den Punkt enthält) nie größer als 1 wird.


\section*{Aufgabe 3}

\subsection*{(a)}
\begin{center}
	\includegraphics*[scale = 0.5]{question3a1.png}
	\includegraphics*[scale = 0.5]{question3a2.png}
\end{center}
Die linke Graphik zeigt den Plot mit $\mu$ = [0 0], die rechte den Plot mit $\mu$ = [-2 2].\\
$\mu$ verschiebt den Graph. Dabei verschiebt der erste Wert den Graph um die x-Achse und der zweite um die y-Achse. Somit ist das Maximum bei (-2,2) und nicht mehr bei (0,0) wie bei der linken Graphik.

\subsection*{(b)} %TODO Formulierung ueberarbeiten?

covarianceX1X2 = 0.7:
\begin{center}
	\includegraphics*[scale = 0.5]{question3b1.png}
\end{center}
covarianceX1X2 = 0.99:
\begin{center}
	\includegraphics*[scale = 0.5]{question3b2.png}
\end{center}
covarianceX1X2 = -0.7:
\begin{center}
	\includegraphics*[scale = 0.5]{question3b3.png}
\end{center}
covarianceX1X2 = -0.99:
\begin{center}
	\includegraphics*[scale = 0.5]{question3b4.png}
\end{center}
Indem man covarianceX1X2 auf eine Zahl im Intervall (0,1) erhöht, gibt es eine Streckung des Graphen anhand einer Ebene senkrecht zur XY-Ebene, im 45$^\{\circ\}$ Winkel zur x - Achse durch den Ursprung. Des Weiteren wird der Graph gestaucht anhand einer Ebene, die senkrecht zu der vorher genannten, durch den Ursprung verläuft.\\
Für Zahlen im Intervall (-1,0) ist es analog, nur dass Streckung und Stauchung miteinander vertauscht ist.\\

\subsection*{(c)}
\begin{itemize}
	\item Kovarianz gibt den absoluten Wert an, wie stark  zwei Zufallszahlen zusammenhängen
	\item Korrelation gibt den relativen Wert an
	\item Korrelation ist standardisiert in einem Intervall [-1,1], während Kovarianz Werte im Intervall ($+\inf, -\inf$) annehmen kann. 
	\item Korrelation ist dimensionslos, während Kovarianz die Einheit der beiden Variablen annimmt.
\end{itemize}

\section*{Aufgabe 4}

\subsection*{(b)}

\begin{center}
	\includegraphics*[scale = 0.5]{question4b.png}
\end{center}

\subsection*{(c)}

\begin{center}
	\includegraphics*[scale = 0.5]{question4c.png}
\end{center}

\subsection*{(d)}

\begin{center}
	\includegraphics*[scale = 0.5]{question4d.png}
\end{center}
\subsection*{(e)}
siehe code.
\subsection*{(f)} %TODO vermutlich falsch?
%Fuer genauere Zufallszahlen waere es sinnvoll, wenn man diejenige Zahl aussucht, welche naeher an der generierten uniformierten Zufallszahl liegt. \\
%Wenn man zum Beispiel die Zufallszahl 0.61 generiert, wuerde der Algorithmus aus (e) 7 auswaehlen, obwohl 6 mit 0.6 naeher liegen wuerde. 
Die Funktion wird unpräzise, weil immer das untere Ende der Wahrscheinlichkeit für einen Wert nicht stimmt. Liegt beispielsweise die Wk für $\leq 9$ bei 0.9, so wird beim Wert 0.9 schon eine 10 generiert, keine 9. Das ist für die meisten Zahlen unproblematisch, weil für jede Zahl der Effekt am unteren Ende der Wahrscheinlichkeit eintritt. Also z.B. wird die Wahrscheinlichkeit für 9 wieder richtig, weil wiederum der oberste Wert für 8 noch zur Wahrscheinlichkeit von 9 gezählt wird. Bei der kleinsten Zahl geschieht dies allerdings nicht. 1 erhält keine ``zusätzliche Wahrscheinlichkeit''. Man kann diesen Fehler beheben, indem man einen zustätzlichen if-Fall einfügt, der abfängt, wenn die zufällig generierte Zahl \textbf{genau} einem F-Wert entspricht und dann direkt den zugehörigen Index als Ergebnis ausgeben.
\end{document}
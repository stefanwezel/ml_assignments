%Template
% !TeX spellcheck = de 
\documentclass[a4paper]{scrartcl}
\usepackage[utf8]{inputenc}
%\usepackage[ngerman]{babel}
\usepackage{geometry,forloop,fancyhdr,fancybox,lastpage}
\usepackage{listings}
\lstset{frame=tb,
	language=Java,
	aboveskip=3mm,
	belowskip=3mm,
	showstringspaces=false,
	columns=flexible,
	basicstyle={\small\ttfamily},
	numbers=left,
	numberstyle=\tiny\color{gray},
	keywordstyle=\color{blue},
	commentstyle=\color{dkgreen},
	stringstyle=\color{mauve},
	breaklines=true,
	breakatwhitespace=true,
	tabsize=3
}
\geometry{a4paper,left=3cm, right=3cm, top=3cm, bottom=3cm}
% Diese Daten müssen pro Blatt angepasst werden:
\newcommand{\NUMBER}{6}
\newcommand{\EXERCISES}{3}
% Diese Daten müssen einmalig pro Vorlesung angepasst werden:
\newcommand{\COURSE}{Einführung in Maschinelles Lernen}
\newcommand{\TUTOR}{TBD}
\newcommand{\STUDENTA}{Maria Heitmeier}
\newcommand{\STUDENTB}{Gwent Krause}
\newcommand{\STUDENTC}{Stefan Wezel}
\newcommand{\DEADLINE}{07.06.2018}




%Math
\usepackage{amsmath,amssymb,tabularx}

%Figures
\usepackage{graphicx,tikz,color,float}
\graphicspath{ {home/stefan/picures/} }
\usetikzlibrary{shapes,trees}

%Algorithms
\usepackage[ruled,linesnumbered]{algorithm2e}

%Kopf- und Fußzeile
\pagestyle {fancy}
\fancyhead[L]{\COURSE}
\fancyhead[C]{\STUDENTA, \STUDENTB, \STUDENTC\\}
\fancyhead[R]{\today}

\fancyfoot[L]{}
\fancyfoot[C]{}
\fancyfoot[R]{Seite \thepage}

%Formatierung der Überschrift, hier nichts ändern
\def\header#1#2{
	\begin{center}
		{\Large\bf Übungsblatt #1}\\
		{(Abgabetermin #2)}
	\end{center}
}

%Definition der Punktetabelle, hier nichts ändern
\newcounter{punktelistectr}
\newcounter{punkte}
\newcommand{\punkteliste}[2]{%
	\setcounter{punkte}{#2}%
	\addtocounter{punkte}{-#1}%
	\stepcounter{punkte}%<-- also punkte = m-n+1 = Anzahl Spalten[1]
	\begin{center}%
		\begin{tabularx}{\linewidth}[]{@{}*{\thepunkte}{>{\centering\arraybackslash} X|}@{}>{\centering\arraybackslash}X}
			\forloop{punktelistectr}{#1}{\value{punktelistectr} < #2 } %
			{%
				\thepunktelistectr &
			}
			#2 &  $\Sigma$ \\
			\hline
			\forloop{punktelistectr}{#1}{\value{punktelistectr} < #2 } %
			{%
				&	
			} &\\
			\forloop{punktelistectr}{#1}{\value{punktelistectr} < #2 } %
			{%
				&
			} &\\
		\end{tabularx}
	\end{center}
}

\begin{document}
	
	
\section*{Aufgabe 1}
\subsection*{(a)}
Bei Klassifikation wird anhand von Features einem Sample eine Kategorie zugeteilt. Bei Regression hingegen wird eine Eigenschaft anhand von anderen (korrelierenden) Eigenschaften eines Samples vorausgesagt.

\subsection*{(b)}
Es handelt sich um ein Regressionsproblem, da ein Feature (Groesse) eines Samples(Mensch) anhand von korrellierenden Features vorhergesagt.

\subsection*{(c)}
Da man dem Sample (Person) eine Kategorie (Geschlecht) zuteilt handelt es sich um ein Klassifizierungsproblem.


\subsection*{(d)}
Beim Supervised Learning benutzt man Daten mit Labels und versucht dann ueblicherweise diese dann fuer Daten mit unbekannten Labels vorauszusagen.\\
Beim Unsupervised Learning haben die Daten keine Labels und man versucht stattdessen in verschiedene Kategorien einzuteilen (bspw. durch Clustering).

\subsection*{(e)}
\subsubsection*{(i)}
Eine Zufallsvariable ist eine Funktion die aus einer Menge von moeglichen Ergebnissen eines Zufallsexperiment auf eine Wahrscheinlichkeit (Maß) abbildet. Hat man eine endliche Augangsmenge von moeglichen Ergebnissen (zb. bei einem Wuefel) hat man eine diskrete Zufallsvariable. Ist die Grundmenge nicht abzaehlbar (zb. Koerpergroesse) dann ist die Zufallsvariable ein Intervall aus einer kontinuirlichen Menge.

\subsubsection*{(ii)}
Eine Wahrscheinlichkeitsfunktion nimmt als Parameter eine diskrete Zufallsvariable an und gibt einen Wert zwischen 0 und 1 aus.\\
Bei der Wahrscheinlickeitsdichtefunktion hingengen ist die Wahrscheinlichkeit für einen bestimmten Wert immer 0 und es lässt sich lediglich die Wahrscheinlichkeit für Intervalle bestimmen. Diese sind dann das Integral der Dichtefunktion welches als Grenzen die obere und untere Schranke des Intervalls hat.

\subsubsection*{(iii)} 
Der Erwartungswert ist der Wert den eine Zufallsvariable nach unendlich vielen Durchführungen eines Zufallsexperiments im Mittel annimmt.

\subsubsection*{(iv)} 
Die Varianz einer Verteilung beschreibt die mittlere quadratische Abweichung zum Erwartungswert. Die Standardabweichung ist die positive Wurzel aus der Varianz. Beide beschreiben also die "Streuung" der Daten um einen durchschnittlichen Wert.

\subsubsection*{(v)} %TODO Formulierung ueberarbeiten.
Die Kovarianz zweier Zufallsvariablen beschreibt wie stark diese linear zusammenhängen. Wird beispielsweise Variable $X$ größer und $Y$ ebenfalls, so spricht man von positiver Kovarianz. (Fall $COV(X,Y)> 0$). Ist das Gegenteil der Fall, also wenn $X$ größer wird und $Y$ kleiner, so spricht man von negativer Kovarianz(Fall $COV(X,Y) < 0$). Wenn $COV(X,Y)=0$, dann sind die Variablen unanbhängig.



\subsection*{(f)}
Die Multiplikation zweier Wahrscheinlichkeitsdichten liefert wieder eine Wahrscheinlichkeitsdichte. Wahrscheinlichkeitsdichten lassen sich als Funktion beschreiben, die:
\begin{itemize}
	\item niemals kleiner 0 wird.
	\item unter deren Kurve die Fläche 1 ist (und damit kontinuierlich ist.)
\end{itemize}
Multipliziert man nun zwei Wahrscheinlichkeitsdichten ist das Produkt wieder eine Funktion, deren Werte niemals kleiner 0 werden.


\section*{Aufgabe 2}

\section*{Aufgabe 3}



\end{document}
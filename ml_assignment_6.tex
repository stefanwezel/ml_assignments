%Template
% !TeX spellcheck = de 
\documentclass[a4paper]{scrartcl}
\usepackage[utf8]{inputenc}
%\usepackage[ngerman]{babel}
\usepackage{geometry,forloop,fancyhdr,fancybox,lastpage}
\usepackage{listings}
\lstset{frame=tb,
	language=Java,
	aboveskip=3mm,
	belowskip=3mm,
	showstringspaces=false,
	columns=flexible,
	basicstyle={\small\ttfamily},
	numbers=left,
	numberstyle=\tiny\color{gray},
	keywordstyle=\color{blue},
	commentstyle=\color{dkgreen},
	stringstyle=\color{mauve},
	breaklines=true,
	breakatwhitespace=true,
	tabsize=3
}
\geometry{a4paper,left=3cm, right=3cm, top=3cm, bottom=3cm}
% Diese Daten müssen pro Blatt angepasst werden:
\newcommand{\NUMBER}{6}
\newcommand{\EXERCISES}{3}
% Diese Daten müssen einmalig pro Vorlesung angepasst werden:
\newcommand{\COURSE}{Einführung in Maschinelles Lernen}
\newcommand{\TUTOR}{TBD}
\newcommand{\STUDENTA}{Maria Heitmeier}
\newcommand{\STUDENTB}{Gwent Krause}
\newcommand{\STUDENTC}{Stefan Wezel}
\newcommand{\DEADLINE}{07.06.2018}




%Math
\usepackage{amsmath,amssymb,tabularx}

%Figures
\usepackage{graphicx,tikz,color,float}
\graphicspath{ {home/stefan/picures/} }
\usetikzlibrary{shapes,trees}

%Algorithms
\usepackage[ruled,linesnumbered]{algorithm2e}

%Kopf- und Fußzeile
\pagestyle {fancy}
\fancyhead[L]{\COURSE}
\fancyhead[C]{\STUDENTA, \STUDENTB, \STUDENTC\\}
\fancyhead[R]{\today}

\fancyfoot[L]{}
\fancyfoot[C]{}
\fancyfoot[R]{Seite \thepage}

%Formatierung der Überschrift, hier nichts ändern
\def\header#1#2{
	\begin{center}
		{\Large\bf Übungsblatt #1}\\
		{(Abgabetermin #2)}
	\end{center}
}

%Definition der Punktetabelle, hier nichts ändern
\newcounter{punktelistectr}
\newcounter{punkte}
\newcommand{\punkteliste}[2]{%
	\setcounter{punkte}{#2}%
	\addtocounter{punkte}{-#1}%
	\stepcounter{punkte}%<-- also punkte = m-n+1 = Anzahl Spalten[1]
	\begin{center}%
		\begin{tabularx}{\linewidth}[]{@{}*{\thepunkte}{>{\centering\arraybackslash} X|}@{}>{\centering\arraybackslash}X}
			\forloop{punktelistectr}{#1}{\value{punktelistectr} < #2 } %
			{%
				\thepunktelistectr &
			}
			#2 &  $\Sigma$ \\
			\hline
			\forloop{punktelistectr}{#1}{\value{punktelistectr} < #2 } %
			{%
				&	
			} &\\
			\forloop{punktelistectr}{#1}{\value{punktelistectr} < #2 } %
			{%
				&
			} &\\
		\end{tabularx}
	\end{center}
}

\begin{document}
\section*{Aufgabe 2}
\subsection*{(a)}
Lol keine Ahnung ayayayayaa
\begin{itemize}
	\item \textbf{(i)} Die Wahrscheinlichkeit $k$ von $n$ Stichproben in einer gegebenen Region zu finden lässt sich lässt sich sich durch eine Binomialverteilung abhängig von $n$ und $p(x)$ beschreiben.\\
	Der Erwartungswert für $k$ lässt sich daher einfach mit $n \cdot P$ berchnen. Stellen wir diese Gleichung um, ergibt sich $P = n \cdot k$ was dann einen Schätzer darstellt, da die Genauigkeit des Erwartungswerts von der Menge an Samples abhängt und noch kein genauer Wert ist.
	
	\item \textbf{(ii)} 
	
\end{itemize}




\subsection*{(b)}
Bei der Parzenfenster Methode kann die Fenstergröße durch eine Funktion $\phi$ bestimmt werden. Mithilfe der Funktion kann man dann die Kantenlänge des Hypercubes welcher das Fenster darstellt bestimmen.\\
Dies ist allerdings nicht immer so einfach. Die k-nearest-neighbor Methode umgeht dieses Problem indem sie mit einem klein genügenden Fenster beginnt, um es dann zu vergrößern, so lange bis die namensgebenden $k$ Samples enthalten sind.


\subsection*{(c)}
Es werden nicht parametrische Wahrscheinlichkeitsdichtefunktionen geschätzt. Also keine Dichten, die einer gewissen Familie anhören und nicht einfach durch die typischen Parameter wie Varianz, Standartabweichung oder Mittelwert geschätzt werden können.\\
Da die beiden Verfahren nicht parametrische Wahrscheinlichkeitsdichten schätzen werden sie auch dementsprechend benannt.



\end{document}
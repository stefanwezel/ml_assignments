%Template
% !TeX spellcheck = de 
\documentclass[a4paper]{scrartcl}
\usepackage[utf8]{inputenc}
%\usepackage[ngerman]{babel}
\usepackage{geometry,forloop,fancyhdr,fancybox,lastpage}
\usepackage{listings}
\lstset{frame=tb,
	language=Java,
	aboveskip=3mm,
	belowskip=3mm,
	showstringspaces=false,
	columns=flexible,
	basicstyle={\small\ttfamily},
	numbers=left,
	numberstyle=\tiny\color{gray},
	keywordstyle=\color{blue},
	commentstyle=\color{dkgreen},
	stringstyle=\color{mauve},
	breaklines=true,
	breakatwhitespace=true,
	tabsize=3
}
\geometry{a4paper,left=3cm, right=3cm, top=3cm, bottom=3cm}
% Diese Daten müssen pro Blatt angepasst werden:
\newcommand{\NUMBER}{2}
\newcommand{\EXERCISES}{4}
% Diese Daten müssen einmalig pro Vorlesung angepasst werden:
\newcommand{\COURSE}{Einführung in Maschinelles Lernen}
\newcommand{\TUTOR}{TBD}
\newcommand{\STUDENTA}{Maria Heitmeier}
\newcommand{\STUDENTB}{Gwent Krause}
\newcommand{\STUDENTC}{Stefan Wezel}
\newcommand{\DEADLINE}{23.05.2018}




%Math
\usepackage{amsmath,amssymb,tabularx}

%Figures
\usepackage{graphicx,tikz,color,float}
\graphicspath{ {home/stefan/picures/} }
\usetikzlibrary{shapes,trees}

%Algorithms
\usepackage[ruled,linesnumbered]{algorithm2e}

%Kopf- und Fußzeile
\pagestyle {fancy}
\fancyhead[L]{\COURSE}
\fancyhead[C]{\STUDENTA, \STUDENTB, \STUDENTC\\}
\fancyhead[R]{\today}

\fancyfoot[L]{}
\fancyfoot[C]{}
\fancyfoot[R]{Seite \thepage}

%Formatierung der Überschrift, hier nichts ändern
\def\header#1#2{
	\begin{center}
		{\Large\bf Übungsblatt #1}\\
		{(Abgabetermin #2)}
	\end{center}
}

%Definition der Punktetabelle, hier nichts ändern
\newcounter{punktelistectr}
\newcounter{punkte}
\newcommand{\punkteliste}[2]{%
	\setcounter{punkte}{#2}%
	\addtocounter{punkte}{-#1}%
	\stepcounter{punkte}%<-- also punkte = m-n+1 = Anzahl Spalten[1]
	\begin{center}%
		\begin{tabularx}{\linewidth}[]{@{}*{\thepunkte}{>{\centering\arraybackslash} X|}@{}>{\centering\arraybackslash}X}
			\forloop{punktelistectr}{#1}{\value{punktelistectr} < #2 } %
			{%
				\thepunktelistectr &
			}
			#2 &  $\Sigma$ \\
			\hline
			\forloop{punktelistectr}{#1}{\value{punktelistectr} < #2 } %
			{%
				&	
			} &\\
			\forloop{punktelistectr}{#1}{\value{punktelistectr} < #2 } %
			{%
				&
			} &\\
		\end{tabularx}
	\end{center}
}

\begin{document}
	
\section*{Aufgabe 1}

\section*{Aufgabe 2}

\section*{Aufgabe 3}

\section*{Aufgabe 4}
\begin{itemize}
	\item[b)]
	\item[c)] Setosa:\\
	\begin{tabular}{c|c|c|c|c}
		&sepale Länge & sepale Breite & petale Länge & petale Breite\\
		\hline
		Mittelwert & 5.0425 &   3.4575 &   1.4650  &  0.2525 \\
		Varianz & 0.3601 &   0.3928  &  0.1875  &  0.1109\\
	\end{tabular}\\ \\
Versicolor:\\
\begin{tabular}{c|c|c|c|c}
	&sepale Länge & sepale Breite & petale Länge & petale Breite\\
	\hline
	Mittelwert & 5.8950 &   2.7450  &  4.2325  &  1.3125 \\
	Varianz & 0.4517  &  0.3063  &  0.4676  &  0.2040\\
\end{tabular}\\ \\
Virginica:\\
\begin{tabular}{c|c|c|c|c}
	&sepale Länge & sepale Breite & petale Länge & petale Breite\\
	\hline
	Mittelwert & 6.5925  &  2.9825  &  5.4975  &  2.0225\\
	Varianz & 0.5989 &   0.3226  &  0.5332  &  0.2741\\
\end{tabular}
	\item[d)] Siehe code
	\item[e)] Die Ergebnisse sind in folgender Tabelle zusammengefasst:\\ \\ \begin{tabular}{c|c|c|c|c}
		Art & true-positives & true-negatives & false-positives & false-negatives\\
		\hline
		Setosa & 10 & 20 & 0 & 0\\
		Versicolor & 10 & 19 & 1 & 0\\
		Virginica & 9 & 20 & 0& 1\\
	\end{tabular}
\end{itemize}


\end{document}
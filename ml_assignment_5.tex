%Template
% !TeX spellcheck = de 
\documentclass[a4paper]{scrartcl}
\usepackage[utf8]{inputenc}
%\usepackage[ngerman]{babel}
\usepackage{geometry,forloop,fancyhdr,fancybox,lastpage}
\usepackage{listings}
\lstset{frame=tb,
	language=Java,
	aboveskip=3mm,
	belowskip=3mm,
	showstringspaces=false,
	columns=flexible,
	basicstyle={\small\ttfamily},
	numbers=left,
	numberstyle=\tiny\color{gray},
	keywordstyle=\color{blue},
	commentstyle=\color{dkgreen},
	stringstyle=\color{mauve},
	breaklines=true,
	breakatwhitespace=true,
	tabsize=3
}
\geometry{a4paper,left=3cm, right=3cm, top=3cm, bottom=3cm}
% Diese Daten müssen pro Blatt angepasst werden:
\newcommand{\NUMBER}{6}
\newcommand{\EXERCISES}{3}
% Diese Daten müssen einmalig pro Vorlesung angepasst werden:
\newcommand{\COURSE}{Einführung in Maschinelles Lernen}
\newcommand{\TUTOR}{TBD}
\newcommand{\STUDENTA}{Maria Heitmeier}
\newcommand{\STUDENTB}{Gwent Krause}
\newcommand{\STUDENTC}{Stefan Wezel}
\newcommand{\DEADLINE}{07.06.2018}




%Math
\usepackage{amsmath,amssymb,tabularx}

%Figures
\usepackage{graphicx,tikz,color,float}
\graphicspath{ {home/stefan/picures/} }
\usetikzlibrary{shapes,trees}

%Algorithms
\usepackage[ruled,linesnumbered]{algorithm2e}

%Kopf- und Fußzeile
\pagestyle {fancy}
\fancyhead[L]{\COURSE}
\fancyhead[C]{\STUDENTA, \STUDENTB, \STUDENTC\\}
\fancyhead[R]{\today}

\fancyfoot[L]{}
\fancyfoot[C]{}
\fancyfoot[R]{Seite \thepage}

%Formatierung der Überschrift, hier nichts ändern
\def\header#1#2{
	\begin{center}
		{\Large\bf Übungsblatt #1}\\
		{(Abgabetermin #2)}
	\end{center}
}

%Definition der Punktetabelle, hier nichts ändern
\newcounter{punktelistectr}
\newcounter{punkte}
\newcommand{\punkteliste}[2]{%
	\setcounter{punkte}{#2}%
	\addtocounter{punkte}{-#1}%
	\stepcounter{punkte}%<-- also punkte = m-n+1 = Anzahl Spalten[1]
	\begin{center}%
		\begin{tabularx}{\linewidth}[]{@{}*{\thepunkte}{>{\centering\arraybackslash} X|}@{}>{\centering\arraybackslash}X}
			\forloop{punktelistectr}{#1}{\value{punktelistectr} < #2 } %
			{%
				\thepunktelistectr &
			}
			#2 &  $\Sigma$ \\
			\hline
			\forloop{punktelistectr}{#1}{\value{punktelistectr} < #2 } %
			{%
				&	
			} &\\
			\forloop{punktelistectr}{#1}{\value{punktelistectr} < #2 } %
			{%
				&
			} &\\
		\end{tabularx}
	\end{center}
}

\begin{document}
\section*{Aufgabe 1}
\subsection*{(a)}
Knoten in einem Bayes Netz stehen für Zufallsvariablen. Die Kanten stehen für Abhängigkeiten. Sind Knoten miteinander verbunden, beinflusst eine Variable die Andere. Den Knoten sind jeweils Wahrscheinlichkeitswerte zugewiesen.


\subsection*{(b)}



\subsection*{(c)}




\subsection*{(d)}
Die Ordnung eines Hidden Markov Models beschreibt um wie viel Zustände zurückgegangen wird um die Übergangswahrscheinlichkeit zu berechnen. 
%TODO evtl ausführlicher

\subsection*{(e)}
$\Omega$ bezeichnet die Menge aller Zustände eines HMM.
Die Menge $V$ behinhaltet die Visible States eines HMM.

\subsection*{(f)}
Die Matrix A beinhaltet die Übergangswahrscheinlichkeiten $P(\omega_j(t+1)|\omega_i(t))=a_{ij}$.\\
Die Matrix B beinhaltet die Übergangswahrscheinlichkieten in sichtbare Zustände gegeben die aktuellen Zustände.


\subsection*{(g)}
\begin{itemize}
	\item \textbf{Bewertungsproblem} Wie wird der Nachfolgezustand anhand der Übergangswahrscheinlichkeiten bestimmt. %TODO nicht sicher
	
	\item \textbf{Dekodierungsproblem} Wie kann man die wahrscheinlichste Folge von hidden states herausfinden, gegeben die Menge aus sichtbaren Zuständen.
	
	\item \textbf{Lernproblem}
\end{itemize}



\end{document}